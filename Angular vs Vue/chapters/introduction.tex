\chapter{Introduction}
\label{cha:Introduction}


\section{Motivation}
The increasing importance of web applications in the modern digital landscape motivates this analysis. According to the UNCTAD Digital Economy Report (2019), businesses and services are rapidly moving online, which increases the demand for efficient, user-friendly, and powerful web applications. The ability of web applications to operate across various platforms allows developers to serve a wide user base without the need to develop separate applications for each operating system or device~\cite{unctad}.

\section{Challenges}

%\subsection{Hardware Platforms}
Web application development necessitates compatibility across multiple hardware platforms, including desktop computers, laptops, tablets, and mobile devices. Mao (2014) identifies the primary challenge as maintaining consistent performance and efficiency across these diverse devices~\cite{mao2014developing}.

%\subsection{Frameworks}
Additionally, the selection and management of development frameworks significantly influence the efficiency and scalability of web applications. Verma (2022) points out that different frameworks exhibit unique strengths and weaknesses concerning performance, maintainability, and user-friendliness. Verma (2022) also stresses the importance of evaluation methods in assessing the performance, security, and user-friendliness of web applications. These evaluations typically include benchmarks, test scenarios, and other empirical methods aimed at verifying the quality of the application~\cite{verma2022comparison}.

\section{Goals}
This analysis aims to provide a brief overview of both frameworks and serve as a decision-making guide by offering valuable insights into the strengths and weaknesses of both technologies. 
These insights assist development teams in making informed decisions about the most suitable technologies for specific application requirements. Additionally, they help identify optimization opportunities to continuously enhance user experience and improve development efficiency.

\section{Structure}
Besides the structure, the motivation section also covers the challenges and goals of this study. The following three chapters include the theoretical background, which discusses web frameworks and compares Vue and Angular, a case study detailing a practical application and its analysis, and concludes with recommendations and future outlook.
%ToDo:
% - Theoretischer Hintergrund
%    - Was sind Web-Frameworks
%    - Warum sind sie wichtig
%    - Einführung in JavaScript-Frameworks: Allgemeine Informationen zu JavaScriptFrameworks
%    - Kurze Beschreibung Angular.js: Ursprung, Entwicklung, Hauptmerkmale
%    - Kurze Beschreibung Vue.js: Ursprung, Entwicklung, Hauptmerkmale
%    - Was sind die Unterschiede zwischen Angular.js und Vue.js
% - Grundlage Vergleichskriterien --> Welche sind Sinnvoll und Messbar -->!!! Benutzerfreundlichkeit und Lernkurve fallen weg weil zu individuell
%    - Performance --> Ladezeiten, Reaktionszeiten, etc...
%    - Entwicklerfreundlichkeit --> Einfachheit der Verwendung, Dokumentation, etc...
%    - Community-Support und Ökosystem --> Anzahl und Aktivität der Community-Mitglieder, Verfügbarkeit und Qualität von Bibliotheken und Tools
% - Fallstudie - Beschreibung der Applikation
%       - Fallstudie 1: Angular.js Applikation
%       - Fallstudie 2: Vue.js Applikation
% - Ergebnisse Vergleichskriterien
%     -Leistung
%           - Benchmark-Tests und reale Beispiele, Vor- und Nachteile in verschiedenen Anwendungskriterien
%     -Community-Support und Ökosystem
%           - Anzahl und Aktivität der Community-Mitglieder
%           - Verfügbarkeit und Qualität von Bibliotheken und Tools
% - Fazit und Empfehlungen
%       - Zusammenfassung der Ergebnisse
%       - Empfehlungen für die Auswahl des geeigneten Frameworks
%       - Zukungsaussichten -> wie könnten sich die Frameworks weiterentwickeln